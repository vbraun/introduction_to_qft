\documentclass[12pt]{article}
\pdfoutput=1

\newcommand{\VersionInformation}{}  % overwritten by Debug.tex
\InputIfFileExists{Debug}{}{}


\usepackage[utf8]{inputenc}
\usepackage{amsmath}
\usepackage{amsthm}
\usepackage{amsfonts}          % if you want the fonts
\usepackage{amssymb}           % if you want extra symbols
%\usepackage{euscript}
\usepackage[mathscr]{eucal}
%\usepackage{stmaryrd}          % \lightning is defined there
\usepackage{graphicx}
\usepackage{color}
%\usepackage{floatflt}
%\usepackage{bbold}
%\usepackage{ulem}
\usepackage[nosort]{cite}
\usepackage{hypbmsec}
\usepackage{fancyvrb}
\usepackage{sepnum}
\usepackage{xspace}
\usepackage{booktabs}
\usepackage{rotating}
\usepackage{multirow}
\usepackage[vcentermath]{youngtab}
%\usepackage{slashbox}
\usepackage{simplewick}
\usepackage[isu,bf]{caption}
\setlength{\captionmargin}{1cm}
\renewcommand{\captionfont}{\small\itshape}


%\usepackage{algorithm}
%\usepackage{algorithmic}

\font\csc=cmcsc10

\newlength{\xtrawidth}
\setlength{\xtrawidth}{8mm}
\newlength{\xtraheight}
\setlength{\xtraheight}{10mm}
\addtolength{\textwidth}{\xtrawidth}
\addtolength{\textwidth}{\xtrawidth}
\addtolength{\oddsidemargin}{-\xtrawidth}
\addtolength{\evensidemargin}{-\xtrawidth}
\addtolength{\textheight}{\xtraheight}
\addtolength{\textheight}{\xtraheight}
\addtolength{\topmargin}{-\xtraheight}


\usepackage[all]{xy}           % Commutative diagrams



\ifx\NoBackreferences\EMPTYMACRO
  % arXiv will complain about option clash in hyperref
  % \usepackage[backref,linktocpage,bookmarks]{hyperref}
  \usepackage{hyperref}
\else
  \usepackage[linktocpage,bookmarks]{hyperref}
\fi

% \usepackage[sort&compress]{natbib}


\def\footnoteautorefname{Footnote}%
\def\itemautorefname{Item}%
\def\sectionautorefname{Section}%
\def\subsectionautorefname{Subsection}%
\def\subsubsectionautorefname{Subsubsection}%
\def\paragraphautorefname{Paragraph}%
\def\subparagraphautorefname{Subparagraph}%
\def\FancyVerbLineautorefname{Line}%

%%%%%%%%%%%%%%% volker's defs %%%%%%%%%%%%%%%

% \entrymodifiers={[F.]}
%\usepackage[floats,textmath,displaymath,delayed,sections,auctex]{preview}

\ifx\DEBUG\EMPTYMACRO
  % Don't use showlabels
\else
  \usepackage{showlabels}
\fi


\newcommand{\lrstack}[3][t]{
  \ensuremath{
    \begin{array}[#1]{c}
      \multicolumn{1}{l}{\displaystyle{#2}\quad} \\[0.5em] 
      \multicolumn{1}{r}{\displaystyle\quad{#3}}
    \end{array}
  }}
\def\clap#1{\hbox to 0pt{\hss#1\hss}}
\def\mathllap{\mathpalette\mathllapinternal}
\def\mathrlap{\mathpalette\mathrlapinternal}
\def\mathclap{\mathpalette\mathclapinternal}
\def\mathllapinternal#1#2{%
\llap{$\mathsurround=0pt#1{#2}$}}
\def\mathrlapinternal#1#2{%
\rlap{$\mathsurround=0pt#1{#2}$}}
\def\mathclapinternal#1#2{%
\clap{$\mathsurround=0pt#1{#2}$}}	

\makeatletter
  \def\adots{\mathinner{\mkern2mu\raise\p@\hbox{.}
      \mkern2mu\raise4\p@\hbox{.}\mkern1mu
      \raise7\p@\vbox{\kern7\p@\hbox{.}}\mkern1mu}}
\makeatother

\newcommand{\comma}[1]{\ensuremath{\sepnum{{.}}{{,}}{}{#1}}}


\newcommand{\eqdef}{%
  \mathrel{\lower.1mm
    \hbox{$\stackrel{\lower.424ex\hbox{\scriptsize def}}{=}$}}
}
%\newcommand{\eqdef}{\stackrel{\mathrm{def}}{=}}
%\newcommand{\eqdef}{=}
\newcommand{\Q}{\ensuremath{{\mathbb{Q}}}}
\newcommand{\R}{\ensuremath{{\mathbb{R}}}}
\newcommand{\C}{\ensuremath{{\mathbb{C}}}}
\newcommand{\Hbb}{\ensuremath{{\mathbb{H}}}}
\newcommand{\Z}{\mathbb{Z}}
% \newcommand{\CP}{\ensuremath{\mathop{\mathbb{C}{\rm P}}}\nolimits}
\newcommand{\CP}{{\ensuremath{\mathop{\null {\mathbb{P}}}\nolimits}}}
\newcommand{\IP}{{\ensuremath{\mathop{\null {\mathbb{P}}}\nolimits}}}
\newcommand{\RP}{\ensuremath{\mathop{\mathbb{R}{\rm P}}}\nolimits}
\newcommand{\F}{\ensuremath{{\mathbb{F}}}}

\newcommand{\ibar}{\ensuremath{{\bar{\text{\it\i\/}}}}}
\newcommand{\jbar}{\ensuremath{{\bar{\text{\it\j\/}}}}}
\newcommand{\alphabar}{\ensuremath{{\bar{\alpha}}}}
\newcommand{\betabar}{\ensuremath{{\bar{\beta}}}}
\newcommand{\sbar}{\ensuremath{{\bar{s}}}}
\newcommand{\zbar}{\ensuremath{{\bar{z}}}}

\newcommand{\FS}{\ensuremath{{\text{FS}}}}

\newcommand{\Dbrane}{D-brane}
\newcommand{\Dbranes}{D-branes}
\newcommand{\YM}{Yang-Mills}
\newcommand{\MW}{Mordell-Weil}
\newcommand{\MWgrp}{\MW{} group}
\newcommand{\MWlat}{\MW{} lattice}
\newcommand{\even}{\ensuremath{\mathrm{ev}}}
\newcommand{\odd}{\ensuremath{\mathrm{odd}}}
\newcommand{\Ktheory}{K-theory}
\newcommand{\Ktheories}{K-theories}
\newcommand{\Kgroup}{K-group}
\newcommand{\Kgroups}{K-groups}
\newcommand{\KRtheory}{KR-theory}
\newcommand{\KRtheories}{KR-theories}
\newcommand{\KRgroup}{KR-group}
\newcommand{\KRgroups}{KR-groups}
\newcommand{\KOtheory}{KO-theory}
\newcommand{\Ktilde}{\widetilde{K}}
\newcommand{\KOtilde}{\widetilde{KO}}
\newcommand{\Kan}{K_\mathrm{an}}
\newcommand{\Kcoh}{K_\mathrm{coh}}
\newcommand{\Kalg}{K_\mathrm{alg}}
\newcommand{\Cech}{{\v{C}ech}}
\newcommand{\Hcech}{{\check{H}}}
\newcommand{\Hdr}{H_\mathrm{DR}}
\newcommand{\Hol}{\mathrm{Hol}}
\newcommand{\Kunneth}{K{\"u}nneth}
\newcommand{\inv}{\mathrm{inv}}

\newcommand{\Ncal}{\mathcal{N}}
\newcommand{\tKop}[1]{\ensuremath{\vphantom{K}^{#1}\!K}}
\newcommand{\tK}{\ensuremath{\tKop{t}}}
\newcommand{\T}[1]{\lbrack{#1}\rbrack} % shift functor X[i]
\newcommand{\ptset}{\ensuremath{\{\text{pt.}\}}}
\newcommand{\iunit}{\ensuremath{\mathrm{i}}}
\newcommand{\Cunits}{\ensuremath{\C^\times}}
\newcommand{\free}{\ensuremath{\text{free}}}
\newcommand{\tors}{\ensuremath{\text{tors}}}

\newcommand{\Moduli}{\mathcal{M}}
\newcommand{\rt}{\ensuremath{\tilde{r}}}

\DeclareMathOperator{\Div}{Div}
\DeclareMathOperator{\TDiv}{TDiv}


\DeclareMathOperator{\diff}{d\!}
\DeclareMathOperator{\Span}{span}
\DeclareMathOperator{\Pic}{Pic}
\DeclareMathOperator{\re}{re}
\DeclareMathOperator{\im}{im}
\DeclareMathOperator{\Tr}{Tr}
\DeclareMathOperator{\tr}{tr}
\DeclareMathOperator{\Mat}{Mat}
\DeclareMathOperator{\Id}{id}
\DeclareMathOperator{\rank}{rank}
\DeclareMathOperator{\codim}{codim}
\DeclareMathOperator{\End}{End}
\DeclareMathOperator{\Aut}{Aut}
\DeclareMathOperator{\Sym}{Sym}
\DeclareMathOperator{\Alt}{Alt}
\DeclareMathOperator{\idx}{Index}
\DeclareMathOperator{\img}{img}
\DeclareMathOperator{\coker}{coker}

\DeclareMathOperator{\Hom}{Hom}
\DeclareMathOperator{\Tor}{Tor}
\DeclareMathOperator{\Ext}{Ext}

\DeclareMathOperator{\HOM}{\underline{Hom}}
\DeclareMathOperator{\TOR}{\underline{Tor}}
\DeclareMathOperator{\EXT}{\underline{Ext}}

\DeclareMathOperator{\Sing}{Sing}
\DeclareMathOperator{\Li}{Li}
\DeclareMathOperator{\Vol}{Vol}
\DeclareMathOperator{\dVol}{dVol}
\DeclareMathOperator{\diag}{diag}

\DeclareMathOperator{\Ind}{Ind}
\DeclareMathOperator{\Res}{Res}
\DeclareMathOperator{\AltInd}{AltInd}
\DeclareMathOperator{\SymInd}{SymInd}
\DeclareMathOperator{\GrInd}{GrInd}

\DeclareMathOperator{\Stab}{Stab}

\newcommand{\Spin}{{\mathop{\text{\textit{Spin}}}\nolimits}}
\newcommand{\sofrak}{{\mathop{\mathfrak{so}}\nolimits}}
\newcommand{\sufrak}{{\mathop{\mathfrak{su}}\nolimits}}
\newcommand{\hfrak}{{\mathop{\mathfrak{h}}\nolimits}}
\newcommand{\soTenC}{{\mathop{\sofrak(10)_\C}\nolimits}}
\newcommand{\Rep}[1]{\ensuremath{\mathbf{\underline{#1}}}}
\newcommand{\barRep}[1]{\ensuremath{\overline{\Rep{#1}}}}
\DeclareMathOperator{\Reg}{Reg}
\DeclareMathOperator{\Ad}{ad}


\newcommand{\textdef}[1]{\textit{#1}}

\newcommand{\Xt}{{\ensuremath{\widetilde{X}}}}
\newcommand{\Vt}{{\ensuremath{\widetilde{V}}}}
\newcommand{\Xb}{{\ensuremath{\overline{X}}}}
\newcommand{\Ct}{{\ensuremath{\widetilde{C}}}}
\newcommand{\ZZZ}{\ensuremath{{\Z_3\times\Z_3}}}
\newcommand{\Lsheaf}{\ensuremath{\mathscr{L}}}
\newcommand{\Osheaf}{\ensuremath{\mathscr{O}}}
\newcommand{\OsheafXt}{\ensuremath{\mathscr{O}_{\Xt}}}
\newcommand{\OsheafBone}{\ensuremath{\mathscr{O}_{B_1}}}
\newcommand{\OsheafBtwo}{\ensuremath{\mathscr{O}_{B_2}}}
\newcommand{\OsheafP}{\ensuremath{\mathscr{O}_{\CP^1}}}
\newcommand{\Vsheaf}{\ensuremath{\mathscr{V}}}
\newcommand{\Wsheaf}{\ensuremath{\mathscr{W}}}
\newcommand{\Esheaf}{\ensuremath{\mathscr{E}}}
\newcommand{\Fsheaf}{\ensuremath{\mathscr{F}}}
\newcommand{\Hsheaf}{\ensuremath{\mathscr{H}}}
\newcommand{\Hsheafdual}{\ensuremath{\mathscr{H}^\vee}}
\newcommand{\Ksheaf}{\ensuremath{\mathscr{K}}}
\newcommand{\Ssheaf}{\ensuremath{\mathscr{S}}}
\newcommand{\dual}{\ensuremath{\vee}}
\newcommand{\Asheaf}{\ensuremath{\mathscr{A}}}
\newcommand{\Bsheaf}{\ensuremath{\mathscr{B}}}
\newcommand{\Csheaf}{\ensuremath{\mathscr{C}}}
\newcommand{\Qsheaf}{\ensuremath{\mathscr{Q}}}
\newcommand{\At}{{\ensuremath{\widetilde{A}}}}
\newcommand{\Ab}{{\ensuremath{\overline{A}}}}

\newcommand{\Yt}{{\ensuremath{\widetilde{Y}}}}

\newcommand{\Lss}{Leray spectral sequence}
\newcommand{\LSss}{Leray-Serre spectral sequence}
\newcommand{\CLss}{Cartan-Leray spectral sequence}

\newcommand{\Htate}{\ensuremath{\widehat{H}}}
\newcommand{\CPambient}{\ensuremath{\CP^2\times \CP^1 \times \CP^2}}
\newcommand{\dP}[1]{\ensuremath{dP_{#1}}}
\newcommand{\FPtimes}{\underline{\times}}
\newcommand{\Xhat}{{\ensuremath{\Hat{X}}}}
\newcommand{\Bhat}{{\ensuremath{\Hat{B}}}}

\newcommand{\Fprepotential}{\mathscr{F}}
\newcommand{\FprepotentialNP}{\mathscr{F}^\text{np}}
\newcommand{\Fprepot}[1]{\ensuremath{\Fprepotential_{{#1},0}}}
\newcommand{\FprepotNP}[1]{\ensuremath{\Fprepot{#1}^\text{np}}}
\newcommand{\FprepotX}{\Fprepot{X}}
\newcommand{\FprepotXNP}{\FprepotNP{X}}
\newcommand{\FprepotXt}{\Fprepot{\Xt}}
\newcommand{\FprepotXtNP}{\FprepotNP{\Xt}}

\newcommand{\Kahler}{K\"ahler\xspace}
\newcommand{\Kcone}{\ensuremath{\mathcal{K}}}

\newcommand{\ThetaEeight}{\ensuremath{\Theta_{E_8}}}
\newcommand{\mathemph}[1]{\textcolor{red}{\mbox{\boldmath $#1$}}}
\newcommand{\isorightarrow}{\ensuremath{\stackrel{\sim}{\rightarrow}}}
\newcommand{\isolongrightarrow}{\ensuremath{\stackrel{\sim}{\longrightarrow}}}
\newcommand{\hooklongrightarrow}{\lhook\joinrel\longrightarrow}
\newcommand{\tmod}{~\mathrm{mod}~}

\newcommand{\cyclperm}{\ensuremath{(\text{cyc})}}
\newcommand{\abbar}{\ensuremath{\alpha\bar{\beta}}}
\newcommand{\Qt}{{\ensuremath{\widetilde{Q}}}}
\newcommand{\QtF}{{\ensuremath{\widetilde{Q}}_F}}
\newcommand{\QtFsub}[1]{{\ensuremath{\widetilde{Q}}_{F,{#1}}}}

\newcommand{\Pt}{{\ensuremath{\widetilde{P}}}}
\newcommand{\Rt}{{\ensuremath{\widetilde{R}}}}

\newcommand{\CY}{\text{CY}}
\newcommand{\Npoints}{\ensuremath{N_p}}

\newcommand{\Etilde}{\ensuremath{\widetilde{E}}}

\newcommand{\ESixAffineLabels}[7]{
  \ensuremath{
    \vcenter{\xymatrix@C=8mm@R=3mm@!0{
        & & & #4 \ar@{-}[dl] & #2 \ar@{-}[l] \\
        #1 \ar@{-}[r] & #3 \ar@{-}[r] & #5 \\
        & & & #6 \ar@{-}[ul] & #7 \ar@{-}[l] 
      }}
  }
}

% \newcommand{\smallESixAffineLabels}[7]{
%   \small
%   \ensuremath{
%     \vcenter{\xymatrix@C=7mm@R=1.5mm@!0{
%         & & & #4 \ar@{-}[dl] & #2 \ar@{-}[l] \\
%         #1 \ar@{-}[r] & #3 \ar@{-}[r] & #5 \\
%         & & & #6 \ar@{-}[ul] & #7 \ar@{-}[l] 
%       }}
%   }
% }

\newcommand{\smallESixAffineLabels}[7]{
  \tiny
  \ensuremath{
    \vcenter{\xymatrix@C=5mm@R=.8mm@!0{
        & & & #4 \ar@{-}[dl] & #2 \ar@{-}[l] \\
        #1 \ar@{-}[r] & #3 \ar@{-}[r] & #5 \\
        & & & #6 \ar@{-}[ul] & #7 \ar@{-}[l] 
      }}
  }
}

\newcommand{\IIA}{\ensuremath{\text{II\!A}}}
\newcommand{\IIB}{\ensuremath{\text{IIB}}}

\newcommand{\Omegabar}{\overline{\Omega}}
\DeclareMathOperator{\Ric}{Ric}
\DeclareMathOperator{\Her}{Her}


\DeclareMathOperator{\conv}{conv}
\DeclareMathOperator{\ord}{ord}

\newcommand{\cone}[1]{\ensuremath{\left<#1\right>}}



\newcommand{\CC}{C\nolinebreak\hspace{-.05em}\raisebox{.4ex}{\tiny\bf +}\nolinebreak\hspace{-.10em}\raisebox{.4ex}{\tiny\bf +}}
\newcommand{\blitzpp}{blitz\nolinebreak\hspace{-.05em}\raisebox{.4ex}{\tiny\bf +}\nolinebreak\hspace{-.10em}\raisebox{.4ex}{\tiny\bf +}}

\newenvironment{descriptionlist}{%
\begin{list}%
{}%
{\renewcommand{\makelabel}[1]{##1}}}%
{\end{list}}

% an array with displaystyle (=large) cells
\newenvironment{displayarray}%
{\everymath{\displaystyle\everymath{}}\array}%
{\endarray}

\newtheorem{theorem}{Theorem}
\newtheorem{lemma}{Lemma}
\newtheorem{example}{Example}
\newtheorem{exercise}{Exercise}
\newtheorem{notation}{Notation}
\newtheorem{proposition}{Proposition}
\newtheorem{fact}{Fact}
\newtheorem{corollary}{Corollary}
\newtheorem{conjecture}{Conjecture}
\newtheorem{definition}{Definition}



\input cyracc.def %sha
\newfont{\twelvecyr}{wncyr10 at 12pt}
%\font\tencyr=wncyr10
%\def\sha{\text{\tencyr\cyracc{Sh}}}
\def\sha{\text{\twelvecyr\cyracc{Sh}}}


%%%%%%%%%%%%%%%%%%%%%%%%%%%%%%%%%%%%%%%%%%%%%


\usepackage{color}
\usepackage{listings} 
\usepackage{courier}

\definecolor{dbluecolor}{rgb}{0.01,0.02,0.7}
\definecolor{dgreencolor}{rgb}{0.2,0.4,0.0}
\definecolor{dgraycolor}{rgb}{0.30,0.3,0.30}

\lstdefinelanguage{SageInputLanguage}{
  language=Python, 
  morekeywords={False,sage,True},
  sensitive=true,
}

\lstdefinestyle{SageInput}{
  language=SageInputLanguage,
  basicstyle=\fontsize{11pt}{11pt}\ttfamily\bfseries,
  commentstyle={\ttfamily\color{dgreencolor}},
  stringstyle={\color{dgraycolor}\bfseries},
  keywordstyle=\ttfamily\color{dbluecolor}\bfseries\color{red},
  xleftmargin=25pt,
  belowskip=3pt,
}

\lstdefinelanguage{SageOutputLanguage}{
  morekeywords={False,True},
  sensitive=true,
}

\lstdefinestyle{SageOutput}{
  language=SageOutputLanguage,
  basicstyle={\fontsize{11pt}{11pt}\ttfamily},
  commentstyle={\ttfamily\color{dgreencolor}},
  keywordstyle={\ttfamily\color{dbluecolor}},
  stringstyle={\ttfamily\color{dgraycolor}},
  xleftmargin=25pt,
  aboveskip=0pt,
}

\lstdefinestyle{DefaultSageInputOutput}{
  identifierstyle=,
  numbersep=5pt,
  aboveskip=0pt,
  belowskip=0pt,
  breaklines=true,
  numberstyle=\footnotesize,
  numbers=right,
}

\newcommand{\textsage}[1]{\lstinline[style=SageInput]{#1}}



%%% Local Variables:
%%% TeX-master: "Main"
%%% eval: (TeX-PDF-mode 1)
%%% End:



\begin{document}
%%%%%%%%%%%%%%%%%%%%%[ Title Page ]%%%%%%%%%%%%%%%%%%%%%%%%%%
\begin{titlepage}
  \vspace*{-2cm}
  \VersionInformation
  \hfill
  \parbox[c]{5cm}{
    \begin{flushright}
%      arXiv:yymm.nnnn [hep-th]
    \end{flushright}
  }
  \vspace*{2cm}
  \begin{center}
    \Huge 
    Introduction to Quantum Field Theory
  \end{center}
  \vspace*{8mm}
  \begin{center}
    \begin{minipage}{\textwidth}
      \begin{center}
        \sc 
        Volker Braun
      \end{center}
      \begin{center}
        \textit{
          University of Oxford, Andrew Wiles Building\\
          Radcliffe Observatory Quarter, Woodstock Road\\
          Oxford, OX2 6GG, United Kingdom
        }
      \end{center}
    \end{minipage}
  \end{center}
  \vspace*{\stretch1}
  \begin{abstract}
    Notes for the introduction to quantum field theory.
  \end{abstract}
  \vspace*{\stretch1}
  \begin{minipage}{\textwidth}
    \underline{\hspace{5cm}}\\
    Email: \texttt{volker.braun@maths.ox.ac.uk}
  \end{minipage}
\end{titlepage}
\tableofcontents
\listoffigures 	% to produce list of figures
\listoftables 	% to produce list of tables



%%% Local Variables:
%%% TeX-master: "Main"
%%% eval: (TeX-PDF-mode 1)
%%% End:



% http://www.phy.olemiss.edu/~luca/Topics/ft/scalar.html

\section{Syllabus}

\subsection{Introduction}

\subsubsection{Fields}

The subject of these lectures is quantum field theory, one of the most
important tools for quantitatively describe physical processes in the
microcosmos. The fundamental objects, you might have guessed it, are
\emph{fields}. They are a function of space and time, and take values
in some set. For the most part of this introduction, we will only
consider real-valued fields
\begin{equation}
  \phi(\vec x, t) \in \R
  \quad \forall (\vec x, t) \in \R^{3,1}  
\end{equation}
What we commonly call particles are excitations of this underlying
physical field. Clearly this has the potential to explain both
particle-like and wave-like behavior, provided we can define
observables and find rules for the time evolution of states.



\subsubsection{Quantum Mechanics}





\subsubsection{What Is Missing in Quantum Mechanics}

\begin{itemize}
\item Decay of excited atoms
\item Particle creation/annihilation
\item $g-2$ of the electron
\end{itemize}


\subsubsection{Spin-0 and the Lorentz Group}

\begin{itemize}
\item Lorentz group
\item Particles = irreducible representations
\item Indistinguishable
\end{itemize}





\subsection{The Path Integral}







\begin{enumerate}
\item 
\item 
\item 
\item Free scalar field
\end{enumerate}


\subsection{{\boldmath $\phi^4$} Theory}

\begin{enumerate}
\item Quartic Interaction
\item Perturbation Theory
\item Spontaneous Symmetry Breaking
\item Lattice
\item Renormalization
\item Solitons in $\phi^4$ theory
\end{enumerate}



\subsection{Ising Model}


\subsection{Fermions}


\newpage


\section{First class (13.10)}

The Schrödinger equation $i\hbar\frac{\partial}{\partial
  t}|\psi,t\rangle = H|\psi,t\rangle$ is not relativistic since it
treats space and time differently. Solution: Make space a parameter
like time.
\begin{definition}
  An operator $\varphi(x,t)$ is a quantum field.
\end{definition}
Quantum mechanics is not a good description for
multi-particle systems where the total number is not
conserved. E.g. excited hydrogen atoms really decay (by creating
photons), so the electron wave function can't be a stationary
solution.

\subsection{Lightning review of electrodynamics}

\begin{equation}
  F_{\mu\nu} =
  \begin{pmatrix}
    0 & E_x & E_y & E_z \\
    -E_x & 0 & -B_z & B_y \\
    - E_y & B_z & 0 & -B_x \\
    -E_z & -B_y & B_x & 0 \\
  \end{pmatrix}
  = \partial_{[\mu} A_{\nu]}
\end{equation}
Maxwell equations $\partial_{[\alpha} F_{\beta\gamma]} = 0$,
$\partial_\mu F^{\nu\mu}=j^\nu$ are equations of motion of action
\begin{equation}
  S = \int d^4x \left[- \tfrac{1}{4} F_{\mu\nu} F^{\mu\nu} + A_\nu j^\nu\right]
\end{equation}

\begin{example}
  Point charge solution located at origin has 4-vector
  potential
  \begin{equation}
    A = \left(
      -\frac{1}{\sqrt{x^2+y^2+z^2}}, 0, 0, 0)
    \right)
  \end{equation}
  We checked that this satisfies equations of motion with
  $j=(-\delta^3(\vec{x}), 0, 0, 0)$. The action integral diverges.
\end{example}
Since there is no magnetic field (in this coordinate frame), the
current-independent part of the Lagrangian density $L=- \tfrac{1}{4}
F_{\mu\nu} F^{\mu\nu} = |B|^2-|E|^2$ diverges just like the energy
stored in the electromagnetic field $|E|^2 + |B|^2$. The action
integral diverges for the same reason that the energy in the electric
field is infinite. 

These infinities are unavoidable in a field theory. Quantum field
theory cannot avoid them either, but has a formalism for dealing with
them.


\section{Second class (15.10)}

\subsection{Klein-Gordon Equation}

What would the relativistic Hamiltonian be? We know the
non-relativistic Hamiltonian for free particle is
$\frac{1}{2m}P^2$. We also know the relativistic energy of a free
particle with 4-momentum $p_\mu = (E/c, p_x, p_y, p_z) = (E/c,
\vec{p})$ is
\begin{equation}
  |p|^2 = p_\mu p^\mu = -m^2 c^2
  \quad \Leftrightarrow \quad
  E = mc^2 \sqrt{1 + \frac{\vec{p}^2}{m^2 c^2}} =
  mc^2 + \frac{1}{2m} \vec{p}^2 
  + O(\tfrac{1}{c}).
\end{equation}
The terms in the Taylor series are rest energy, the non-relativistic
energy, and higher-order corrections. This suggests the ``relativistic
Schr\"odinger'' equation
\begin{equation}
  -i\hbar \frac{\partial}{\partial t} |\phi, t\rangle =
  H |\phi, t\rangle =
  \sqrt{m^2c^4 + c^2 P^2}
  H |\phi, t\rangle
\end{equation}
but the square root makes it difficult to make sense out of that
equation. In particular, the series expansion in $P$ will mean
infinite number of terms. It is also not Lorentz invariant, still
treating time and space differently. Square (i.e.~apply twice) the
operator on both sides and switch to position space,
$P_j=i\hbar \partial_j$:
\begin{equation}
  \begin{gathered}
    -\hbar^2 \frac{\partial^2}{\partial t^2} \phi(x, t) 
    =
    \left(
      m^2c^4  - \hbar^2 c^2 \sum_{j=1}^3 \frac{\partial}{\partial x_j}
    \right) \phi(x,t)
    \\
    \Leftrightarrow \quad
    \left(
      \partial_\mu \partial^\mu  -
      \frac{m^2 c^2}{\hbar^2}
    \right) \phi(x,t) = 0
  \end{gathered}
\end{equation}
This is the Klein-Gordon equation. It is manifestly relativistic, but
second order in time derivatives. So initial conditions involve
$\phi(x,0)$ and $\dot\phi(x,0)$. Depending on initial conditions,
$\int d^3x |\phi(x,t)|^2$ is not conserved.

Instead of going to second derivatives, one might hope that there
would be a way to do it with first derivatives in space and time. But
that is incompatible with the action principle. For example, consider
this simple 1-dimensional theory (so we don't have to figure out how
to contract the $\partial_\mu$):
\begin{equation}
  S = \int dx \; \phi \partial \phi
  = \int dx \; \partial\phi^2 - \int dx \; (\partial \phi) \phi
  = [\phi^2]_{-\infty}^\infty - \int dx \; \phi \partial \phi
\end{equation}
with sensible boundary conditions for $\phi$ the surface term will
vanish. But then we get $S=-S \Rightarrow S=0$. There is one way to
evade this conclusion, if $\phi$ \emph{anti-commutes} then the last
sign is reversed and the action is not constrained. This will be
important later on when we talk about fermions.


\subsection{Action Principle}

The Klein-Gordon equation is the equation of motion for the Lagrangian
($c=\hbar=1$ from now on)
\begin{equation}
  \mathcal{L} = -\frac{1}{2} 
  \partial_\mu \phi \partial^\mu \phi
  - \frac{1}{2} m^2 \phi^2
\end{equation}
Check by setting the variation to zero:
\begin{equation}
  0 = \delta S = \int d^4x \delta\mathcal{L} =
  -\int d^4x \partial_\mu (\delta \phi \partial^\mu \phi)
  + \int d^4x \delta \phi (\partial_\mu \partial^\mu - m^2) \phi
\end{equation}
The first summand is a surface term. We only allow compactly-supported
variations $\delta \phi$, hence the surface term is zero. The second
term vanishes precisely when $\phi$ satisfies the Klein-Gordon
equation, which are the equations of motion for the Lagrangian.

Another Lagrangian would be 
\begin{equation}
  \mathcal{L}' = \frac{1}{2} \phi \partial_\mu \partial^\mu \phi
  - \frac{1}{2}m^2\phi^2
\end{equation}
The difference is a divergence
$\mathcal{L}'-\mathcal{L}=\partial_\mu(\frac{1}{2}\phi\partial^\mu
\phi)$. If the field at infinity is sufficiently well-behaved then the
surface term vanishes, and the equations of motion are the same. This
is the case for the free scalar field. In general, whether or not
boundary terms contribute can depend on the details of the physical
theory.


\subsection{Canonical Quantization}

By definition, this means 
\begin{itemize}
\item Take the classical Hamiltonian $H(P, Q)$.
\item Promote $P$, $Q$ to operations with $[Q, P] = i$.
\end{itemize}
We reviewed the harmonic oscillator $H=\frac{1}{2}Q^2 + \frac{1}{2}m^2
Q^2$. Trick: there is a linear combination $a$, $a^\dagger$ of $P$ and
$Q$ that act as creation/annihilation operators. In this notation,
\begin{equation}
  H= m(a^\dagger a + \tfrac{1}{2} )
  ,\quad
  [a, a^\dagger] = 1.
\end{equation}
The operator $a$ annihilates the ground state: $a|0\rangle =
0$. Therefore, its energy is $H|0\rangle = \frac{m}{2}|0\rangle$.


\section{Third class (16.10.)}

First homework.

\subsection{Canonical Quantization of Free Scalar}

Recall the Lagrangian density
\begin{equation}
  \mathcal{L} = -\frac{1}{2} 
  \partial_\mu \phi \partial^\mu \phi
  - \frac{1}{2} m^2 \phi^2.
\end{equation}
The canonically conjugate momentum is
\begin{equation}
  \pi(x,t) = \frac{\partial\mathcal{L}}{\partial \dot\phi} =
  \partial_t \phi = \dot\phi
\end{equation}
Hence the Hamiltonian density is 
\begin{equation}
  \mathcal{H} = \pi \dot\phi - \mathcal{L} = 
  \frac{1}{2}\pi^2 + 
  \frac{1}{2}(\nabla \phi)^2 +
  \frac{1}{2} m^2 \phi^2
\end{equation}
Canonical quantization amounts to promoting $\phi$, $\pi$ to operators
and imposing the commutation relations
\begin{equation}
  \begin{gathered}[]
    [\phi(\vec{x},t), \phi(\vec{y},t)] = 0 = 
    [\pi(\vec{x},t), \pi(\vec{y},t)] = 0
    \\
    [\phi(\vec{x}, t), \pi(\vec{y}, t)] = i \delta^3(\vec{x}-\vec{y})
  \end{gathered}
\end{equation}


\subsection{Creation/Annihilation Operators in Momentum Space}

Classical solutions to the Klein-Gordon equation are plane waves
\begin{equation}
  \phi = e^{i\vec{k}\vec{x} \pm i\omega t}
  ,\quad
  \omega = \sqrt{\vec{k}^2 + m^2}.
\end{equation}
Plane waves are a basis for the space of functions. And one that is
well adapted to the free field. Need to talk about the integration in
momentum-space, though. A good (Lorentz-invariant) measure for the
3-dimensional spacelike slice of momentum space $k_\mu=(\omega, \vec{k})$
is
\begin{equation}
  \mu(k) \sim \int d^4k\; \delta(k^2+m^2) \Theta(k_0)
  = \int d^3\vec{k} \; \frac{1}{2\omega}
\end{equation}
where $\Theta$ is the Heavyside (step) function. A useful abbreviation
for the following is 
\begin{equation}
  \widetilde{dk} = \frac{d^3k}{2\omega (2\pi)^3}.
\end{equation}
Finally, imposing reality of the function $\phi$ relates the Fourier
modes to be
\begin{equation}
  \phi = \int \widetilde{dk} 
  \left[
    a(\vec{k}) e^{ikx} + a^\ast(\vec{k}) e^{-ikx} 
  \right]
\end{equation}
where $\ast$ is complex conjugation and $kx = k_\mu x^\mu$ is the
4-vector inner product. Just as in the harmonic oscillator, we can
invert this to write the Fourier modes as linear combination of field
and conjugate momentum:
\begin{equation}
  a(\vec{k}) = \int d^3x e^{-ikx} (i\dot \phi + \omega \phi)
\end{equation}
A longish algebra exercise yields the Hamiltonian
\begin{equation}
  H = \cdots = \frac{1}{2} \int \widetilde{dk}\; \omega
  \left[
    a^\ast(\vec{k}) a(\vec{k}) + 
    a(\vec{k}) a^\ast(\vec{k}) 
  \right]
\end{equation}
where I have been careful to not commute the Fourier modes $a$,
$a^\ast$. We now perform canonical quantization by promoting $a$,
$a^\ast$ to operators $a$, $a^\dagger$. The canonical commutation
relations become
\begin{equation}
  \begin{gathered}[]
    [a(\vec{k}), a(\vec{k}')] = 0 =
    [a^\dagger(\vec{k}), a^\dagger(\vec{k}')]
    \\
    [a(\vec{k}), a(\vec{k}')] = 
    2 \omega (2\pi)^3 \delta^3(\vec{k}-\vec{k}')
  \end{gathered}
\end{equation}
and the Hamiltonian is
\begin{equation}
  H = \int \widetilde{dk} \; \omega
  \left[
    a^\dagger(\vec{k}) a(\vec{k})
    + \omega (2\pi)^3 \delta^3(0)
  \right]
\end{equation}
The $\delta^3(0)$ looks bad, but, going back into the calculation,
just came from the space-integral $\int d^3x e^{i\vec{k}\vec{x}} =
(2\pi)^3 \delta(\vec{k})$. It just signifies that the zero point
energy density, integrated over all space, is infinite because the
volume of space is infinite. Instead, we should factor off the volume
of space times $V$ times the actual energy density
$\mathcal{E}_0$. Hence
\begin{equation}
  H = \int \widetilde{dk}  \;
  a^\dagger(\vec{k}) a(\vec{k})
  + V  \mathcal{E}_0
  ,\qquad
  \mathcal{E}_0 = \int \widetilde{dk}\; \omega^2.
\end{equation}
The zero point energy \emph{density} still diverges! This is not so
easily fixable. It is perhaps not surprising, if there is a harmonic
oscillator at each point in space-time then there are still infinitely
many in a given volume, contributing an infinite zero-point energy
when you add them up.



\bibliographystyle{utcaps} 
\renewcommand{\refname}{Bibliography}
\addcontentsline{toc}{section}{Bibliography} 
\bibliography{Main}

\end{document}


%%% Local Variables:
%%% eval: (TeX-PDF-mode 1)
%%% End:
